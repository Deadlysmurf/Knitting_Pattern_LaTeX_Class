\documentclass{knitting_pattern}
\usepackage[utf8]{inputenc}


%----------------------------------------------------------------------------------------
%	Information: Information for the title and footer
%----------------------------------------------------------------------------------------
\patternname{Name of the Pattern}

\DesignerName{Designer Name}
\DesignerWebsite{Ravelry.com}
\DesignerEmail{testing\@testing.com}
\DesignerTwitter{Twitter Handle}


%----------------------------------------------------------------------------------------
%	Start the document
%----------------------------------------------------------------------------------------
\begin{document}

%----------------------------------------------------------------------------------------
%	Several photo options: Place them where you would like them to show up.
%----------------------------------------------------------------------------------------
\ThreePhotoPanelNoBox{Hat_3.jpg}{Hat_1.jpg}{Hat_2.jpg}

\TwoPhotoPanelNoBox{Hat_3.jpg}{Hat_2.jpg}

%----------------------------------------------------------------------------------------
%	Several Mixed Text and Photo options: Place them where you would like them to show up.
%----------------------------------------------------------------------------------------

%% Several options for the copy about the pattern. Place the boxes where you would like them to show up.

\PatternAboutLeftPhotoOneBox{Hat_1.jpg}{This is an option that allows you to have a box of text on the right hand side.}

\PatternAboutRightPhotoTwoBoxes{Hat_1.jpg}{This is an option that allows you to have a box of text on the left hand side.}

%----------------------------------------------------------------------------------------
%	Several Text Box Options: Place them where you would like them to show up.
%----------------------------------------------------------------------------------------

\OneColumnBoxNoTitle{This is an option that allows you to have a full sheet sized box of the about text. I like to use it for the pattern about header at the top}

\OneColumnBox{Yarn}{1 skein of [This yarn] (x yds/x m per 3.5oz/100g), or similar yarn.}

\OneColumnBox{Size Info}{It fits heads with circumferences between x cm to x cm (x inches to x inches}

\OneColumnBox{Gauge}{Unstretched Gauge: x sts/x rounds (Stretched Gauge\: x sts/x rounds) \\ 10 cm/4 inches square in pattern stitch}

\TwoColumnBox{Needles}{
\begin{itemize}[noitemsep, leftmargin=0pt]
    \item 3.5 mm (US 4) 16 inch (40 cm) circular needle
    \item 3.5 mm (US 4) DPNs for the crown
\end{itemize}
}{Notions}{
\begin{itemize}[noitemsep, leftmargin=0pt]
	\item Cable needle
	\item Stitch marker
	\item Darning needle
\end{itemize}}

\Instructions{Place the instructions here. Write them all up nicely and clearly so that your tech editor likes it.}

%----------------------------------------------------------------------------------------
%	Charts + Written Instructions: add the chart image file, and instructions
%----------------------------------------------------------------------------------------

%% There are some standard changes that my tech editor wants, so I'm adding notes here.
    %% No capital to start the line
    %% No period at the end of the line
    %% Manually have to add the line breaks
    %% Repeated rows should be reduced if possible
    %% Rnd(s) vs row(s)

\Chart{Sample_Chart.jpg}{
Worked as 10 stitch repeats.\\
Rnds 1 -3: K2, p2, k4, p2
Round 4: K2, p2, 2/2 LC, p2.
}

%----------------------------------------------------------------------------------------
%	Abbreviations: Pick the terms to list, they will list in the order defined in the glossary
%----------------------------------------------------------------------------------------

%% This is the list from the knitting_glossary file - if you add there, please add here.

% Simple Stitches
\glsadd{P}
\glsadd{K}

% Increases
\glsadd{YO}
\glsadd{M1Loop}
\glsadd{M1L}
\glsadd{M1R}

% Decreases
\glsadd{K2tog}
\glsadd{SSK}
\glsadd{P2tog}
\glsadd{SSP}

%Cables
\glsadd{2/2LC}
\glsadd{2/2RC}

\glsadd{2/1LC}
\glsadd{2/1RC}

\glsadd{1/2LC}
\glsadd{1/2RC}

\glsadd{1/1LC}
\glsadd{1/1RC}

% Slipped Stitches
\glsadd{S1}

%----------------------------------------------------------------------------------------
%	Print all the Abbreviations: You can comment this out if you don't want it to appear
%----------------------------------------------------------------------------------------
\Abbreviations{}

%----------------------------------------------------------------------------------------
%	End the document
%----------------------------------------------------------------------------------------
\end{document}
